\section{Paso 1:Descargar SonarQube} 
docker pull sonarqube
\begin{itemize}
    \item  En mi caso ya se descargo el Sonarqube
\end{itemize}
\begin{center}
\includegraphics[width=\columnwidth]{images/1}\newline
\end{center}
\section{Paso 2: Ejecutar una instancia de SonarQube } 
\begin{itemize}
    \item docker run -d --name sonarqube -p 9000:9000 sonarqube
\end{itemize}
\begin{center}
\includegraphics[width=\columnwidth]{images/2}\newline
\end{center}
\section{Paso 3: Ingresar al portal con las credenciales } 
\begin{itemize}
    \item http://localhost:9000/
    \item 192.168.99.100:9000
    \item user: admin
    \item pass:admin
\end{itemize}
\begin{center}
\includegraphics[width=\columnwidth]{images/3}\newline
\end{center}
\section{Paso 4: Crear una nueva aplicación con el nombre aplicacionNetCore } 
\begin{center}
\includegraphics[width=\columnwidth]{images/4}\newline
\end{center}
\section{Paso 5: Generar el token de la nueva aplicación aplicacionNetCore, debera devolver algo similar a 8a15d2a89c8636f15eb32ebee0993b8d16bff94e } 
\begin{center}
\includegraphics[width=\columnwidth]{images/5}\newline
\includegraphics[width=\columnwidth]{images/6}\newline
\includegraphics[width=\columnwidth]{images/7}\newline
\includegraphics[width=\columnwidth]{images/8}\newline
\end{center}
\section{Decargar Net Core e instalar} 

\begin{itemize}
    \item  https://dotnet.microsoft.com/download/dotnet-core/thank-you/sdk-3.1.300-windows-x64-installer
\end{itemize}
\begin{center}
\includegraphics[width=\columnwidth]{images/18}\newline
\end{center}

\section{En un terminal ejecutar e instalar sonar-scanner} 
Ya lo tengo instalado en mi ordenador
\begin{itemize}
    \item  dotnet tool install --global dotnet-sonarscanner
\end{itemize}
\begin{center}
\includegraphics[width=\columnwidth]{images/10}\newline
\end{center}
\section{En un terminal, acceder a una ruta donde creara una nueva aplicacion } 
\begin{itemize}
    \item dotnet new sln -o aplicacionNetCore 
    \item cd aplicacionNetCore 
    \item dotnet new console
    \item dotnet sln aplicacionNetCore.sln add aplicacionNetCore.csproj 
\end{itemize}
\begin{center}
\includegraphics[width=\columnwidth]{images/11}\newline
\includegraphics[width=\columnwidth]{images/12}\newline

\end{center}
\section{ En el mismo terminal, iniciar la sesión de revisión de sonarqube } 
\begin{itemize}
    \item dotnet SonarScanner begin /k:"aplicacionNetCore" /d:sonar.host.url="http://localhost:9000" /d:sonar.login="e7e2acdf4d6e5469cd486b29068f0b5ab9d8ad2e"
\end{itemize}
\begin{center}
\includegraphics[width=\columnwidth]{images/13}\newline
\end{center}
\section{ Compilar la aplicación } 
\begin{itemize}
    \item dotnet build
\end{itemize}
\begin{center}
\includegraphics[width=\columnwidth]{images/14}\newline
\end{center}
\section{ Cerramos la sesión } 
\begin{itemize}
    \item dotnet SonarScanner end /d:sonar.login="e7e2acdf4d6e5469cd486b29068f0b5ab9d8ad2e"
\end{itemize}
\begin{center}
\includegraphics[width=\columnwidth]{images/15}\newline
\includegraphics[width=\columnwidth]{images/16}\newline

\end{center}
\begin{itemize}
    \item Como Resulatado Final
    \begin{center}

\includegraphics[width=\columnwidth]{images/17}\newline
\end{center}
\end{itemize}